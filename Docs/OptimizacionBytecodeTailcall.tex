\documentclass[12pt]{article}
\usepackage[utf8]{inputenc}
\usepackage[T1]{fontenc}
\usepackage[spanish]{babel}
\usepackage{amsmath,amssymb,amsthm}
\usepackage{enumitem}
\usepackage[margin=1.5cm]{geometry}
\usepackage{aligned-overset}
\usepackage{enumitem}

%------------  macros y entornos  -----------------
\newcommand{\length}{\operatorname{length\  }}
\DeclareMathOperator{\longJumpOp}{longJump}
\newcommand{\longJump}[2]{\longJumpOp\ #1\ #2}
\newcommand{\bcc}[1]{\mathsf{bcc}\ #1}
\newcommand{\bctc}[1]{\mathsf{bctc}\ #1}
\newcommand{\subst}[1]{\quad\text{\footnotesize\(\{#1\}\)}}
\newcommand{\concat}{\mathbin{+\!\!+}}   % «+\!\!+» sin huecos extras
\newtheorem{lemma}{Lema}
\newtheorem{theorem}{Teorema}
\theoremstyle{remark}
\newtheorem*{remark}{Observación}
%--------------------------------------------------
\newcommand{\bccp}[1]{\mathsf{bcc}^{\prime}\ #1}
\newcommand{\overeq}[1]{\overset{\mathrm{#1}}{=}}
\newcommand{\overeqa}[1]{\overset{\mathrm{#1}}&{=}}
\newcommand{\overlea}[1]{\overset{\mathrm{#1}}&{\le}}


\begin{document}
Para demostrar que el bytecode generado mediante la optimización de taillcall es efectivamente menor queremos mostrar que 
\[
  \length\!\bigl(\bcc{t}\bigr)\;\le\;
  \length\!\bigl(\bccp{t}\bigr). \qquad  \texttt{t::TTerm}
\]
donde $\bcc{t}$ es el bytecode optimizado y $\bccp{t}$ el original, es decir, previa optimización con taillcall

Antes de demostrar eso demostraremos el siguiente lema:
\begin{lemma}
Para todo $t : \mathrm{TTerm}$ se verifica
\[
  \length\bigl(\bctc{t}\bigr)\;\le\;
  \length\bigl(\bcc{t}\bigr)+1.
\]
\end{lemma}

\begin{proof}
Inducción estructural sobre $t$.  \\

\textbf{Caso base.}  
$t$ construido con \texttt{V} o \texttt{Const}:
\[
  \length\!\bigl(\bctc{t}\bigr)
   \overeq{def.bctc} \length\bigl(\bcc{t}\concat[\texttt{RETURN}]\bigr)
   \overeq{length.(++)} \length\!\bigl(\bcc{t}\bigr)+1 .
\]

\textbf{Caso Inductivo.} 
Sea $t$ un término cualquiera y supongamos la hipótesis inductiva (H.I)
$\length \bctc{t'} \le\length \bcc{t'} +1$
para todo sub‑término $t'\preccurlyeq t$.
Analizamos cada constructor de \texttt{TTerm}:

\begin{enumerate}[label=\textbf{(\alph*)},wide,topsep=4pt]
%--------------------------------------------------
\item \textbf{Aplicación}  
      ($t=\texttt{App}\;\_\;t_1\,t_2$):
\begin{align*}
\length \bctc{t} \overeqa{def.bctc}\ \ \length (\bcc{t_1}\concat\bcc{t_2}\concat[\texttt{RETURN}]) \\
 \overeqa{length.(++)}\length (\bcc{t_1}\concat\bcc{t_2})+1 \\
 \overeqa{def.bcc}\ \ \length(\bcc{t})+1 .
\end{align*}

%--------------------------------------------------
\item \textbf{Condicional cero}  
      ($t=\texttt{IfZ}\;\_\;c\;t_1\;t_2$).  
      Sean $\ell_1=\length(\bctc{t_1})+2$, $\ell_1'=\length(\bcc{t_1})+2$, $\ell_2=\length(\bctc{t_2})$ y $\ell_2'=\length(\bcc{t_2})$:

\begin{align*}
\length(\bctc{t}) \overeqa{def.bctc} \ \ \length\!\bigl(\bcc{c}\concat
    [\texttt{CJUMP},\ell_1]\concat
    \longJump{(\bctc{t_1})}{{\ell_2}}\concat
    [\texttt{JUMP},\ell_2]\concat
    \bctc{t_2}\bigr) \\
 \overeqa{length.(++)} \length(\bcc{c})+2+\length(\bctc{t_1})+2+\length(\bctc{t_2}) \\
 \overlea{H.I.} \quad \  \length(\bcc{c})+2+\bigl(\length(\bcc{t_1})+1\bigr)+2+\bigl(\length(\bcc{t_2})+1\bigr) \\
 \overeqa{length.(++)}  \length\!\bigl(\bcc{c}\concat
    [\texttt{CJUMP},\ell_1']\concat
    \longJump{(\bcc{t_1})}{{\ell_2'}}\concat
    [\texttt{JUMP},\ell_2']\concat
    \bcc{t_2}\bigr) \\
 \overeqa{def.bcc} \ \ \; \length\bigl(\bcc{t}\bigr)+1 
\end{align*}

\item \textbf{Let de descarte}  
    ($t=\texttt{Let}\; i' \; \texttt{``\_''} \;t_1\,(\texttt{Sc1 }t_2)$). 
    Sea $ t_2' = \texttt{varChanger} \; 0 \; \texttt{varLibres} \; \texttt{varBound} \; t_2 $

\begin{align*}
\length(\bctc{t}) \overeqa{def.bctc} \ \ \length\!\bigl(      \bcc{t_1} \concat \texttt{[DISCARD]} \concat \bctc{t_2'}       \bigr)  \\ 
\overeqa{length.(++)} \length (\bcc{t_1}) + \length \texttt{[DISCARD]} + \length (\bctc{t_2'}) \\ 
  \overlea{H.I.} \quad \; \length (\bcc{t_1}) + \length \texttt{[DISCARD]} + \length (\bcc{t_2'}) + 1 \\
  \overeqa{length.(++)} \length\!\bigl(      \bcc{t_1} \concat \texttt{[DISCARD]} \concat \bcc{t_2'}       \bigr) + 1\\
  \overeqa{def.bcc} \quad \length(\bcc{t}) + 1
\end{align*}

%--------------------------------------------------
\item \textbf{Let anidado}  
      ($t=\texttt{Let}\;\_\;\_\;t_1\,(\texttt{Sc1 }t_2)$)\, con subcasos:

\begin{enumerate}[label=\textbf{(\roman*)}, leftmargin=6em]
  \item $t_2=\texttt{Print }\_\;s\;(\texttt{V }\_\;(\texttt{Bound }0))$
  \begin{align*}
      \length(\bctc{t}) \overeqa{def.bctc} \length\bigl(\bctc{(\texttt{Print }\_\;s\;t_1)}\bigr) \\ 
      \overlea{H.I.} \ \ \length\bigl(\bcc{(\texttt{Print }\_\;s\;t_1)}\bigr)+1 \\
      \overeqa{def.bcc} \  \length(\bcc{t})+1 
  \end{align*}
  
  \item $t_2=\texttt{Let }i\;\_\;\_\;(\texttt{Print }\_\;s\;(\texttt{V }\_\;(\texttt{Bound }0)))\,(\texttt{Sc1 }t_3)$
  \begin{align*}
  \length(\bctc{t})
     \overeqa{def.bctc} \ \length\!\bigl(
         \bcc{(\texttt{Print }i\,s\;t_1)}\concat
         [\texttt{SHIFT}]\concat
         \bctc{(\texttt{letSimp }t_3)}
       \bigr) \\
     \overeqa{length.++} \length\!\bigl(\bcc{(\texttt{Print }i\,s\;t_1)}\bigr)+1+
       \length\!\bigl(\bctc{(\texttt{letSimp }t_3)}\bigr)        \\
     \overlea{H.I.} \quad \length\!\bigl(\bcc{(\texttt{Print }i\,s\;t_1)}\bigr)+1+
        \bigl(\length(\bcc{(\texttt{letSimp }t_3)})+1\bigr)      \\
     \overeqa{length.++} \length\!\bigl(
         \bcc{(\texttt{Print }i\,s\;t_1)}\concat
         [\texttt{SHIFT}]\concat
         \bcc{(\texttt{letSimp }t_3)} 
       \bigr) + 1\\
     \overeqa{def.bcc} \ \  \length(\bcc{t})+1
  \end{align*}

  \item $t_2=\texttt{V }\_\;(\texttt{Bound }0)$
  \begin{align*}
        \length(\bctc{t}) \overeqa{def.bctc} \length(\bctc{t_1}) \\ 
        \overlea{H.I.} \ \  \length(\bcc{t_1})+1 \\
        \overeqa{def.bcc} \length(\bcc{t})+1
  \end{align*}


  \item \emph{Resto de subcasos}  
        \begin{align*}
            \length(\bctc{t}) \overeqa{def.bctc} \length(\bcc{t}) \\
            \overlea{} \quad \length(\bcc{t}) + 1
        \end{align*}
\end{enumerate}
\end{enumerate}

\end{proof}

\begin{remark}
Se usa que  
$\length\bigl(\text{longJump }xs\,j\bigr)=\length(xs)$,  
demostrable por inducción en la lista $xs$.
\end{remark}

\begin{theorem}
Para todo $t :: \mathrm{TTerm}$ se cumple
\[
  \length\!\bigl(\bcc{t}\bigr)\;\le\;
  \length\!\bigl(\bccp{t}\bigr).
\]
\end{theorem}

\begin{proof}
Por inducción estructural sobre \texttt{TTerm}.

\paragraph{Caso base.}
Cuando $t$ es construido con \texttt{V} o \texttt{Const}, ambas
implementaciones son idénticas, así que la desigualdad es trivial.

\paragraph{Paso inductivo.}
Supongamos como hipotesis inductiva (H.I.) que
\[
\length\bigl(\bcc{t'}\bigr)\le\length\bigl(\bccp{t'}\bigr)
\]
para todo sub‑término $t'\preccurlyeq t$.
Solo hay que tratar los constructores modificados; los restantes son
idénticos en ambas versiones.

\begin{enumerate}[label=\textbf{(\alph*)},wide,topsep=4pt]
%--------------------------------------------------
\item \textbf{Expresiones lambda}  
      ($t=\texttt{Lam}\;\_\;\_\;\_\;(\texttt{Sc1 }t_1)$)
\begin{align*}
\length\!\bigl(\bcc{t}\bigr)
  \overeqa{def.bcc}
    \quad \length\!\Bigl([\texttt{FUNCTION},\length\!\bigl(\bctc{t_1}\bigr)]
                   \concat\bctc{t_1}\Bigr)\\
  \overeqa{length.(++)} 
    2+\length\!\bigl(\bctc{t_1}\bigr)\\
  \overlea{Lema\,1} \ \ 
    2+\bigl(\length\!\bigl(\bcc{t_1}\bigr)+1\bigr)\\
   \overeqa{} \quad \ 2 + 1 + \length\!\bigl(\bcc{t_1}\bigr) \\
  \overeqa{length.(++)}
    \length\!\Bigl([\texttt{FUNCTION},\length\!\bigl(\bcc{t_1}\bigr)]
                   \concat[\texttt{RETURN}]\Bigr)+
    \length\!\bigl(\bcc{t_1}\bigr)\\
  \overlea{H.I.} \quad \ 
      \length\!\Bigl([\texttt{FUNCTION},\length\!\bigl(\bcc{t_1}\bigr)]
                   \concat[\texttt{RETURN}]\Bigr)+
    \length\!\bigl(\bccp{t_1}\bigr)\\
    \overeqa{} \quad \ \length\!\Bigl([\texttt{FUNCTION},\length\!\bigl(\bcc{t_1}\bigr)]
                   \concat\bccp{t_1}\concat[\texttt{RETURN}]\Bigr)\\
  \overeqa{def.\bccp} \ \ 
    \length\!\bigl(\bccp{t}\bigr).
\end{align*}
%--------------------------------------------------
\item \textbf{Recursión mediante punto fijo}  
      ($t=\texttt{Fix}\;\_\;\_\;\_\;\_\;\_\;(\texttt{Sc2 }t_1)$)
\begin{align*}
\length\!\bigl(\bcc{t}\bigr)
  \overeqa{def.bcc} \ \ 
    \length\!\Bigl([\texttt{FUNCTION},\length\!\bigl(\bctc{t_1}\bigr)]
                   \concat\bctc{t_1}\concat[\texttt{FIX}]\Bigr)\\
  \overeqa{length.(++)}
    2+\length\!\bigl(\bctc{t_1}\bigr)\\
  \overlea{Lema\,1} \ \ \;
    2+\bigl(\length\!\bigl(\bcc{t_1}\bigr)+1\bigr)\\
  \overeqa{length.(++)} 
    \length\!\Bigl([\texttt{FUNCTION},\length\!\bigl(\bcc{t_1}\bigr)]
                   \concat[\texttt{FIX}]
                   \concat[\texttt{RETURN}]\Bigr)+
    \length\!\bigl(\bcc{t_1}\bigr)\\
  \overlea{H.I.} \quad \ 
      \length\!\Bigl([\texttt{FUNCTION},\length\!\bigl(\bcc{t_1}\bigr)]
                   \concat[\texttt{FIX}]
                   \concat[\texttt{RETURN}]\Bigr)+
    \length\!\bigl(\bccp{t_1}\bigr)\\
    \overeqa{}\quad \ \; \length\!\Bigl([\texttt{FUNCTION},\length\!\bigl(\bcc{t_1}\bigr)]
                   \concat\bccp{t_1}
                   \concat[\texttt{FIX}]
                   \concat[\texttt{RETURN}]\Bigr)\\
  \overeqa{def.\bccp} \ \;
    \length\!\bigl(\bccp{t}\bigr).
\end{align*}
\end{enumerate}

Como todos los casos (modificados o no) satisfacen la cota,
la proposición es válida para todo $t$.
\end{proof}

\end{document}
